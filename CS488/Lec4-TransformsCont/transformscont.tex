\documentclass[12pt]{article}

\setlength\parindent{0pt}
\newcommand{\myt}[1]{\textbf{\underline{#1}}}

\usepackage{mathtools}
\usepackage{amssymb}
\usepackage{graphicx}

\title{\vspace{-15ex}CS488 Lecture 4 - Transforms Continued\vspace{-1ex}}
\date{May 10th, 2017}
\author{Graham Cooper}

\begin{document}
	\maketitle
	
	\section*{3D Rotation}
	How can we represent a rotation($\theta$) with respect to an arbitrary axis?\\
	\subsection*{Formulas}
	
	see last slide/pictures\\
	
	\subsection*{Remarks}
	\begin{itemize}
		\item An arbitrary seuence of rotation, translation, reflection, scaling and shear results in an affine transformation
		\item affine transformations preserve parallelism of lines but not angles and lengths
	\end{itemize}

	By convention Right hand is the frame, thumb is x, index finger is y, middle finger is z, and for the view, we use the left hand\\
	
	\section*{Change of Base}
	
	
	
\end{document}
