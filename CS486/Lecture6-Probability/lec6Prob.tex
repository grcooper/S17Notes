\documentclass[12pt]{article}

\setlength\parindent{0pt}
\newcommand{\myt}[1]{\textbf{\underline{#1}}}

\usepackage{mathtools}
\usepackage{amssymb}
\usepackage{graphicx}

\title{\vspace{-15ex}Lecture 8 - More Probability\vspace{-1ex}}
\date{May 24th, 2017}
\author{Graham Cooper}

\begin{document}
	\maketitle
	
	When looking at a Bayes net, see slide 24, lecture 7\\
	The numbers here are just examples\\
	
	\begin{tabular}{c | c}
		P(a) & P(!a) \\ \hline
		0.6 & 0.4
	\end{tabular}\\

	\begin{tabular}{c | c | c}
		A & C & P(C|A) \\ \hline
		a & c & 0.9 \\
		a & !c & 0.1 \\
		!a & c & 0.3 \\
		!a & !c & 0.7 \\
	\end{tabular}\\
	
	\begin{tabular}{c c c | c}
		A & B & D & P(D|A,B) \\ \hline
		a & b & d & \\
		a & b & !d & \\
		a & !b & d & \\
		a & !b & !d & \\
		!a & b & d & \\
		!a & b & !d & \\
		!a & !b & d & \\
		!a & !b & !d & \\	
	\end{tabular}\\

	Can we have arrows going both ways? No we want an acyclic directed graph.\\ Bayesion networks do not have cycles
	
	D-Seperations - Slide 34 lecture 7\\
	1. Independent (first rule) - I missed part of the answer to this :S\\
	2. Dependent if we do not observer flue, Independent if we do (second rule)\\
	3. Dependent if we do not observe fever or flu, independedent if we do\\
	4. Independent if we do not observer fever (rule 3), dependent if we do observe fever (the path opens up)\\
	5. Independed if we do not observe fever (Rule 3), depended if we do observe fever (the path opens up)\\

	Finished on slide 8? Lecture 8

	
\end{document}
