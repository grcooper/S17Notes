\documentclass[12pt]{article}

\setlength\parindent{0pt}
\newcommand{\myt}[1]{\textbf{\underline{#1}}}

\usepackage{mathtools}
\usepackage{amssymb}
\usepackage{graphicx}

\title{\vspace{-15ex}Lec 9 AI - Numpy things \vspace{-1ex}}
\date{May 29th, 2017}
\author{Graham Cooper}

\begin{document}
	\maketitle
	From lecture 8 - slide 12\\
	
	f = np.array([[0.6, 0.4], [0.1, 0.9]);\\
	g = np.array([[0.2, 0.8],[0.3,0.7]]);
	
	\begin{tabular}{c | c c}
		a/b & 0 & 1 \\ \hline
		0 & 0.6 & 0.4 \\
		1 & 0.1, 0.9
	\end{tabular}\\

	\begin{tabular}{c | c c}
		b/c & 0 & 1 \\ \hline
		0 & 0.2 & 0.8 \\
		1 & 0.3 & 0.7
	\end{tabular}\\

	f.reshape(2(A),2(B),1(C))\\
	g.reshape(1(A), 2(B), 2(C))\\
	h = f *g\\
	
	If we want to see the size of the resulting array, then we call this, gives us the size of each dimension\\
	h.shape()\\
	$[2,2,2]$\\
	
	
	We can restrict an array in python
	f[1,:] - We restrict to dimension 1, and all of the values are considered. The next three lines are equivalent\\
	slc = [slice(none)] * 2\\
	slc[0] = 1\\
	f[slc]\\
	
	
	
\end{document}
